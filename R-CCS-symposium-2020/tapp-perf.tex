\documentclass[conference]{IEEEtran}
\usepackage{cite}
\usepackage{amsmath,amssymb,amsfonts}
\usepackage[dvipdfmx]{graphicx}
\usepackage{graphicx}
\usepackage{color}
\usepackage{textcomp}
\usepackage{xcolor}
\definecolor{blue}{rgb}{0.00, 0.00, 1.00}
\usepackage{listings}
\usepackage{datetime2}
\usepackage{algorithm}
\usepackage{algorithmicx}
\usepackage{algpseudocode}
\def\BibTeX{{\rm B\kern-.05em{\sc i\kern-.025em b}\kern-.08em T\kern-.1667em\lower.7ex\hbox{E}\kern-.125emX}}
\lstset{
  language={Fortran},
  basicstyle={\ttfamily},
  identifierstyle={\small},
  %	commentstyle={\smallitshape},
  % keywordstyle={\small\bfseries},
  keywordstyle={\small},
  ndkeywordstyle={\small},
  %	stringstyle={\small\ttfamily},
  frame={tb},
  breaklines=true,
  columns=[l]{fullflexible},
  numbers=left,
  xrightmargin=0zw,
  xleftmargin=2zw,
  numberstyle={\scriptsize},
  stepnumber=1,
  numbersep=1zw,
  lineskip=-0.5ex,
  %	belowcaptionskip=10pt,abovecaptionskip=10pt
}


\begin{document}

\title{
Target applications performance estimation through the codesign of Fugaku computer
}

% double blinded proof reading version
% \if 0
\author{
\IEEEauthorblockN{1\textsuperscript{st} Kazunori Mikami}
\IEEEauthorblockA{\textit{Flagship 2020 Project} \\
\textit{Riken R-CCS}\\
kazunori.mikami@riken.jp}
\and
\IEEEauthorblockN{2\textsuperscript{nd} Hirofumi Tomita}
\IEEEauthorblockA{\textit{Flagship 2020 Project} \\
\textit{Riken R-CCS}\\
htomita@riken.jp}
\and
\IEEEauthorblockN{3\textsuperscript{rd} Soichiro Suzuki}
\IEEEauthorblockA{\textit{Flagship 2020 Project} \\
\textit{Riken R-CCS}\\
soichiro.suzuki@riken.jp}
\and
\IEEEauthorblockN{4\textsuperscript{th} Kazuo Minami}
\IEEEauthorblockA{\textit{Flagship 2020 Project} \\
\textit{Riken R-CCS}\\
minami\_kaz@riken.jp}
}
% double blinded proof reading version
% \fi

\maketitle

\begin{abstract}

The target applications representing the nine social and scientific priority issues
were chosen for codesigning Fugaku computer, a.k.a. post-K.
Each of the nine target applications represents the typical work load in the
originating priority issue.  They cover different numerical schemes,
the different space and time discretization types, the different grid structure,
the different data elements.
Consequently, they show quite different computing characteristics and exhibit
the bottle necks in the system's different components.
Addressing and relaxing the performance bottle necks of the target applications
is expected to help improving the computational performance of the applications,
not only in the priority issues but also in the wide range of high performance
computing demand on Fugaku.
The list of target applications and their objectives, the numerical algorithm,
the computing characteristics and the estimated performance will be shown in the poster.

The target applications undergo many performance analyses by using various tools.
The tools can be categorized, for example, as below:
\begin{itemize}
    \item software simulator executed on non-Fugaku platform.
    \item performance estimation tool based on precise PA data from FX100.
    \item hardware emulator
    \item Fugaku prototype test vehicle
\end{itemize}
There are multiple simulators and estimation tools for supporting different user/usage.
The output example from these tools will be shown in the poster.

For each design and update of the application, detail analysis is
conducted in order to identify the latest computing bottle neck,
and the corresponding feasibility study of system design is requested
to the system development and the manufacturing vendor,
expecting the improved system design parameters.
The updated system design in turn provides the target applications
the opportunity of further optimization from different perspectives.
Thus, after all, the codesign of the target applications and Fugaku computer
is effectively the repeated procedure of mutual optimization
in terms of performance, power and economy.
The design of Fugaku allows the user to choose the processor operating frequency
and the mode of economy. Although the default Fugaku operation should provide
the efficient environment for most of the applications, the users will be able to
explicitly set up the best combination for their applications,
as the target application developers do.
The estimated performance under the restricted power consumption will be
shown in the poster.

The codesign in the early stage can contribute to the design of systems architecture
such as the processor and memory specification and configuration,
and the codesign in the later stage can contribute to the design of system software
such as compilers and libraries.
Ideally, the codesign of Fugaku and the applications can be completed at the same time.
Some codesign achievements reflected to the system will be shown in the poster.


Target applications reference~\cite{tapp-performance:webpage-public}.

\end{abstract}

\begin{IEEEkeywords}
Fugaku computer,
Target application,
Priority issue,
Performance estimation,
Codesign
\end{IEEEkeywords}

%	\section{Introduction}
%	\section{Conclusion}
\bibliographystyle{jplain}
\bibliography{tapp-perf}
\end{document}

